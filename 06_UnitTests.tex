\documentclass[compress]{beamer}%
\usepackage[utf8]{inputenc}
\usetheme{Warsaw}

% \usepackage

\title{Initiation aux frameworks : \emph{JUnit}}
\subtitle{Automatiser les tests unitaires avec JUnit, FestAssert et Mockito}

\author{Thomas Duchatelle (duchatelle.thomas@gmail.com)}
\institute{Capgemini, pour Yves Rocher}

\setbeamertemplate{navigation symbols}{} 
\useoutertheme{infolines}
\setbeamertemplate{footline}
{%
  \leavevmode%
  \hbox{\begin{beamercolorbox}[wd=.5\paperwidth,ht=2.5ex,dp=1.125ex,leftskip=.3cm plus1fill,rightskip=.3cm]{author in head/foot}%
    \usebeamerfont{author in head/foot}\insertshortauthor
  \end{beamercolorbox}%
  \begin{beamercolorbox}[wd=.41\paperwidth,ht=2.5ex,dp=1.125ex,leftskip=.3cm,rightskip=.3cm plus1fil]{title in head/foot}%
    \usebeamerfont{title in head/foot}\insertshorttitle 
  \end{beamercolorbox}%
  \begin{beamercolorbox}[wd=.09\paperwidth,ht=2.5ex,dp=1.125ex,leftskip=.3cm plus1fill,rightskip=.3cm]{author in head/foot}%
    \usebeamerfont{author in head/foot}\insertframenumber/\inserttotalframenumber
  \end{beamercolorbox}}%
  \vskip0pt%
}

\AtBeginSection[]{
  \begin{frame}{Sommaire}
  \small \tableofcontents[currentsection, hideothersubsections]
  \end{frame} 
}

\definecolor{fontcolor}{rgb}{0.92,0.92,0.99}
\usepackage{listings}
\lstset{language=Java, numbers=left, tabsize=2, frame=single, breaklines=true,  numberstyle=\tiny\ttfamily,basicstyle=\small, framexleftmargin=5mm, backgroundcolor=\color{fontcolor}, xleftmargin=5mm, basicstyle=\tiny }

\graphicspath{images}

\begin{document}


% Pages de présentations...
\frame{\titlepage}
  
\section*{Plan}
\frame{\tableofcontents[hideallsubsections]}



%%%%%%%%%%%%%%%%%%%%%%%%%%%%%%%%%%%%%%%%%
%% TEST UNITAIRES
\section{Tests unitaires}

\subsection{Objectifs et définitions}

\begin{frame}{Tests unitaires}
	\framesubtitle{Objectifs et définitions}
	
	\begin{block}{Tests unitaires}
		Isoler une fonctionnalité ou un composant et tester son fonctionnement hors contexte. 
	\end{block}
	
	\pause
	\begin{block}{Dans le cadre de la \emph{Séparation des Préoccupations}}
		Une classe de test pour chaque brique logicielle, testée indépendamment des autres.
	\end{block}
	
\end{frame}

\begin{frame}
	
	Intérêts des tests unitaires :
	\begin{itemize}[<+->]
	\item tester tous les cas possibles d'une briques : passant et non-passant
	\item assurer un non-régression sur les fonctionnalités testées, quelque soit le développeur
	\item ne nécessite pas d'avoir fini l'application pour tester un composant
	\end{itemize}
	
\end{frame}

\subsection{Bonnes pratiques}

\begin{frame}{Bonne pratique}
	\framesubtitle{TDD : Test Driven Development}

	\begin{block}{Développement Piloté par les Tests}
		Méthode de développement consistant à écrire les tests avant de développer le code.
	\end{block}
	
	\pause
	Nouveau développement : 
	\begin{enumerate}
		\item écrire l'interface de la brique à développer
		\item écrire les tests, à partir des spécifications
		\item vérifier que les tests \textbf{ne} passent \textbf{pas}
		\item développer la fonctionnalité
		\item vérifier que les tests passent
	\end{enumerate}
	
	% correctif ?
\end{frame}


%%%%%%%%%%%%%%%%%%%%%%%%%%%%%%%%%%%%%%%%%
%% JUNIT
\section{JUNIT}

\subsection{Framework JUnit}

\begin{frame}{Framework Junit}

	\begin{block}{JUnit}
		JUnit est un \emph{framework} exécutant les tests unitaires d'une application.
	\end{block}
	
	\pause
	\begin{description}
	\item[Framework] JUnit s'occupe de lister les tests à exécuter, les faire passer dans le contexte approprié et collecter les résultats afin d'en fournir un rapport.
	\end{description}
	
	\pause
	\begin{block}{Maven et JUnit}
		Maven, outils de compilation, exécute les tests unitaires à chaque compilation. En cas d'échec, il ne produit pas le binaire.
	\end{block}

\end{frame}


\subsection{Première classe de test}

\begin{frame}[fragile]{Première classe de test}
	\framesubtitle{... avec Spring}
	
	\begin{lstlisting}	
@RunWith(SpringJUnit4ClassRunner.class)
@ContextConfiguration(locations = {
  "classpath:spring/addressrepository-core.xml",
  "classpath:spring/mocked-persistence.xml"
})
public class CalculatriceImplTest {

  @Inject
  private ICalculatrice calculatrice;

  @Test
  public void test_une_fonction() throws Exceptions {
	// mon test
  }
}
	\end{lstlisting}
	
	\pause
	\begin{itemize}[<+->]
	\item \texttt{$@$RunWith} : détermine l'outil à utiliser pour les tests. Ici une extension pour Spring.
	\item \texttt{$@$ContextConfiguration} : liste des fichiers de configuration de Spring
	\item \texttt{$@$Test} : déclare la méthode comme un test à exécuter. La méthode doit-être publique, sans argument ni de retour.
	\end{itemize}

\end{frame}

\subsection{Structure d'une méthode de test}

\begin{frame}[fragile]{Structure d'une méthode de test}

	Une méthode de test comporte 3 parties :
	\begin{enumerate}
	\pause
	\item création du jeu de données
	\pause
	\item exécution du test
	\pause
	\item vérification des résultats
	\end{enumerate}
	
\end{frame}

\begin{frame}[containsverbatim]{Exemple simple}

	\begin{lstlisting}
@Test
public void testAdd() {
  // 1. Initilization
  int a = 6;
  int b = 12;
  
  // 2. Execution
  int result = calculatrice.add(a, b);
  
  // 3. Verification / Assertion
  assertThat(result).isEqualTo(18);
}
	\end{lstlisting}
\end{frame}


\section{Fest Assert}

\subsection{Définition}

\begin{frame}{Fest Assert}
	\framesubtitle{Écrire les assertions dans un langage courant}
	
	\begin{block}{Fest Assert}
		Outils facilitant l'écriture des assertions pour se rapprocher d'un langage courant.
	\end{block}

\end{frame}

\subsection{Assertions basiques}
% test int, string, collection (+extract.from) , exception, objet spécial

\begin{frame}[containsverbatim]{Assertions : types basiques}
	
	\begin{lstlisting}
// types primitifs
assertThat(12 - 9).isEqualTo(3).isGreaterThanOrEqualTo(3).isLessThan(4);

// String
assertThat("Bonjour monde !").isEqualToIgnoringCase("BONJOUR MONDE !).startWith("Bonjour").contains("mon");

// Instance / classe
assertThat(yoda).isInstanceOf(Jedi.class);
assertThat(frodo.getName()).isEqualTo("Frodo");
assertThat(frodo).isNotEqualTo(sauron);


assertThat(frodo).isIn(fellowshipOfTheRing);
assertThat(sauron).isNotIn(fellowshipOfTheRing);
	\end{lstlisting}
\end{frame}


\begin{frame}[containsverbatim]{Assertions : collections}
	
	\begin{lstlisting}
assertThat(frodo).isIn(fellowshipOfTheRing);
assertThat(sauron).isNotIn(fellowshipOfTheRing);

assertThat(fellowshipOfTheRing).hasSize(9)
                               .contains(frodo, sam)
                               .excludes(sauron);
	\end{lstlisting}
\end{frame}


\begin{frame}[containsverbatim]{Assertions : exceptions}
	
	\begin{lstlisting}
try {
  calculatrice.div(42, 0); // argggl !
  // si ArithmeticException n'a pas ete levee, le test echoue avec le message : 
  // "Expected IndexOutOfBoundsException to be thrown"
  failBecauseExceptionWasNotThrown(ArithmeticException.class);
} catch (Exception e) {
  assertThat(e).isInstanceOf(ArithmeticException.class)
               .hasMessageContaining("by zero")
               .hasNoCause();
}
	\end{lstlisting}
\end{frame}



\section{Mockito}

\section{Conclusion}
%Ne remplace pas les tests d'intégration !

\end{document}
































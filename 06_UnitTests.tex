\documentclass[compress]{beamer}%
\usepackage[utf8]{inputenc}
\usetheme{Warsaw}

% \usepackage

\title{Initiation aux frameworks : \emph{JUnit}}
\subtitle{Automatiser les tests unitaires avec JUnit, FestAssert et Mockito}

\author{Thomas Duchatelle (duchatelle.thomas@gmail.com)}
\institute{Capgemini, pour Yves Rocher}

\setbeamertemplate{navigation symbols}{} 
\useoutertheme{infolines}
\setbeamertemplate{footline}
{%
  \leavevmode%
  \hbox{\begin{beamercolorbox}[wd=.5\paperwidth,ht=2.5ex,dp=1.125ex,leftskip=.3cm plus1fill,rightskip=.3cm]{author in head/foot}%
    \usebeamerfont{author in head/foot}\insertshortauthor
  \end{beamercolorbox}%
  \begin{beamercolorbox}[wd=.41\paperwidth,ht=2.5ex,dp=1.125ex,leftskip=.3cm,rightskip=.3cm plus1fil]{title in head/foot}%
    \usebeamerfont{title in head/foot}\insertshorttitle 
  \end{beamercolorbox}%
  \begin{beamercolorbox}[wd=.09\paperwidth,ht=2.5ex,dp=1.125ex,leftskip=.3cm plus1fill,rightskip=.3cm]{author in head/foot}%
    \usebeamerfont{author in head/foot}\insertframenumber/\inserttotalframenumber
  \end{beamercolorbox}}%
  \vskip0pt%
}

\AtBeginSection[]{
  \begin{frame}{Sommaire}
  \small \tableofcontents[currentsection, hideothersubsections]
  \end{frame} 
}

\definecolor{fontcolor}{rgb}{0.92,0.92,0.99}
\usepackage{listings}
\lstset{language=Java, numbers=left, tabsize=2, frame=single, breaklines=true,  numberstyle=\tiny\ttfamily,basicstyle=\small, framexleftmargin=5mm, backgroundcolor=\color{fontcolor}, xleftmargin=5mm, basicstyle=\tiny }

\graphicspath{images}

\begin{document}


% Pages de présentations...
\frame{\titlepage}
  
\section*{Plan}
\frame{\tableofcontents[hideallsubsections]}



%%%%%%%%%%%%%%%%%%%%%%%%%%%%%%%%%%%%%%%%%
%% TEST UNITAIRES
\section{Tests unitaires}

\subsection{Objectifs et définitions}

\begin{frame}{Tests unitaires}
	\framesubtitle{Objectifs et définitions}
	
	\begin{block}{Tests unitaires}
		Isoler une fonctionnalité ou un composant et tester son fonctionnement hors contexte. 
	\end{block}
	
	\pause
	\begin{block}{Dans le cadre de la \emph{Séparation des Préoccupations}}
		Une classe de test pour chaque brique logicielle, testée indépendamment des autres.
	\end{block}
	
\end{frame}

\begin{frame}
	
	Intérêts des tests unitaires :
	\begin{itemize}[<+->]
	\item tester tous les cas possibles d'une briques : passant et non-passant
	\item assurer la non-régression sur les fonctionnalités testées, quelque soit le développeur
	\item ne nécessite pas d'avoir fini l'application pour tester un composant
	\end{itemize}
	
\end{frame}

\subsection{Bonnes pratiques}

\begin{frame}{Bonne pratique}
	\framesubtitle{TDD : Test Driven Development}

	\begin{block}{Développement Piloté par les Tests}
		Méthode de développement consistant à écrire les tests avant de développer le code.
	\end{block}
	
	\pause
	\begin{alertblock}{Correctifs}
		Avant de corriger le code, reproduire l'erreur en test unitaire !
	\end{alertblock}
	
	\pause
	Correctif : 
	\begin{enumerate}[<+->]
		\item ajouter un test mettant en évidence le bug. \emph{Il ne doit pas passer.}
		\item développer le correctif
		\item vérifier que les tests passent (nouveau + non régression)
	\end{enumerate}
	
\end{frame}

\begin{frame}
	Nouveau développement : 
	\begin{enumerate}
		\item écrire l'interface de la brique à développer
		\item écrire les tests, à partir des spécifications
		\item vérifier que les tests \textbf{ne} passent \textbf{pas}
		\item implémenter la fonctionnalité
		\item vérifier que les tests passent
	\end{enumerate}
	

	\pause
	Évolution : 
	\begin{enumerate}
		\item modifier / compléter les tests unitaires
		\item vérifier que les tests \textbf{ne} passent \textbf{pas}
		\item développer l'évolution
		\item vérifier que les tests passent
	\end{enumerate}
	
\end{frame}


%%%%%%%%%%%%%%%%%%%%%%%%%%%%%%%%%%%%%%%%%
%% JUNIT
\section{JUNIT}

\subsection{Framework JUnit}

\begin{frame}{Framework Junit}

	\begin{block}{JUnit}
		JUnit est un \emph{framework} exécutant les tests unitaires d'une application.\\
		\pause
		\begin{itemize}
		\item liste les tests à exécuter
		\item les exécute dans le contexte approprié
		\item collecte les résultats afin d'en fournir un rapport.
		\end{itemize}
	\end{block}
		
	\pause
	\begin{block}{Maven et JUnit}
		Maven, outils de compilation, exécute les tests unitaires à chaque compilation. En cas d'échec, il ne produit pas le binaire.
	\end{block}

\end{frame}


\subsection{Première classe de test}

\begin{frame}[fragile]{Première classe de test}
	\framesubtitle{... avec Spring}
	
	\begin{lstlisting}	
@RunWith(SpringJUnit4ClassRunner.class)
@ContextConfiguration(locations = {
  "classpath:spring/employeerepository-context.xml"
})
public class CalculatriceImplTest {

  @Inject
  private ICalculatrice calculatrice;

  @Test
  public void test_une_fonction() throws Exceptions {
	// mon test
  }
}
	\end{lstlisting}
	
	\pause
	\begin{itemize}[<+->]
	\item \texttt{$@$RunWith} : détermine l'outil à utiliser pour les tests. Ici une extension pour Spring.
	\item \texttt{$@$ContextConfiguration} : liste des fichiers de configuration de Spring
	\item \texttt{$@$Test} : déclare la méthode comme un test à exécuter. La méthode doit-être publique, sans argument ni de retour.
	\end{itemize}

\end{frame}

\subsection{Structure d'une méthode de test}

\begin{frame}{Structure d'une méthode de test}

	Une méthode de test comporte 3 parties :
	\begin{enumerate}[<+->]
	\item création du jeu de données
	\item exécution du test
	\item vérification des résultats
	\end{enumerate}
	
\end{frame}

\begin{frame}[containsverbatim]{Exemple simple}

	\begin{lstlisting}
@Test
public void testAdd() {
  // 1. Initilization
  int a = 6;
  int b = 12;
  
  // 2. Execution
  int result = calculatrice.add(a, b);
  
  // 3. Verification / Assertion
  assertThat(result).isEqualTo(18);
}
	\end{lstlisting}
\end{frame}


\section{Fest Assert}

\subsection{Définition}

\begin{frame}{Fest Assert}
	\framesubtitle{Écrire les assertions dans un langage courant}
	
	\begin{block}{Fest Assert}
		Outils facilitant l'écriture des assertions pour se rapprocher d'un langage courant.
	\end{block}
	
	\pause
	\begin{description}
	\item[Assertion] Condition qui doit être vérifiée pour continuer. Si elle ne l'est pas, le test s'interrompt et il est en échec (\emph{failure}).
	\end{description}

\end{frame}

\subsection{Assertions basiques}
% test int, string, collection (+extract.from) , exception, objet spécial

\begin{frame}[containsverbatim]{Assertions : types basiques}
	
	\begin{lstlisting}
// types primitifs
assertThat(12 - 9).isEqualTo(3)
                  .isGreaterThanOrEqualTo(3)
                  .isLessThan(4);

// String
assertThat(frodo.getName()).isEqualTo("Frodo");
assertThat("Bonjour monde !").isEqualToIgnoringCase("BONJOUR MONDE !")
                             .startWith("Bonjour")
                             .contains("mon");

// Instance / classe
assertThat(yoda).isInstanceOf(Jedi.class);
assertThat(frodo).isNotEqualTo(sauron);
	\end{lstlisting}
\end{frame}


\begin{frame}[containsverbatim]{Assertions : collections}
	
	\begin{lstlisting}
assertThat(frodo).isIn(fellowshipOfTheRing);
assertThat(sauron).isNotIn(fellowshipOfTheRing);

assertThat(fellowshipOfTheRing).hasSize(9)
                               .contains(frodo, sam)
                               .excludes(sauron);
                               

assertThat(extractProperty("age", Integer.class).from(fellowshipOfTheRing)).contains(35, 17);
	\end{lstlisting}
\end{frame}


\begin{frame}[containsverbatim]{Assertions : exceptions}
	
	\begin{lstlisting}
try {
  calculatrice.div(42, 0); // argggl !
  
  // si ArithmeticException n'a pas ete levee, le test echoue avec le message : 
  // "Expected IndexOutOfBoundsException to be thrown"
  failBecauseExceptionWasNotThrown(ArithmeticException.class);
  
} catch (Exception e) {
  assertThat(e).isInstanceOf(ArithmeticException.class)
               .hasMessageContaining("by zero")
               .hasNoCause();
}
	\end{lstlisting}
\end{frame}


\begin{frame}[fragile]{Assertions : objets}
	\framesubtitle{Écrire ses propres assertions}
	
	\begin{lstlisting}
// Objectifs :
assertThat(employee).isHiredBy(yvesRocher)
                    .hasEmail("foo.bar@yrnet.com")
                    .hasEmailDomain("yrnet.com")
                    .isRA()
                    .isRaOf("rc", "vpci");
	\end{lstlisting}
	
\end{frame}

\begin{frame}[fragile]{Assertions : objets}
	\begin{lstlisting}
public class EmployeeAssertion extends AbstractAssert<EmployeeAssertion, Employee> {
  
  /** Constructeur obligatoire */
  public EmployeeAssertion(Employee actual) {
    super(actual, EmployeeAssertion.class);
  }
  
  public EmployeeAssertion isRa() {
    if(actual.getManagedApplications().isEmpty()) {
      throw new AssertionError("Employee is not RA");
    }
    
    return this;
  }
  
  public EmployeeAssertion isHiredBy(Enterprise expected) {
    if( expected != null && ! expected.isEquals(actual.getEnterprise())) {
      throw new AssertionError("Expected enterprise to be " + expected + ", but was " + actual.getEnterprise());
    }
    
    return this;
  }
}
	\end{lstlisting}
	
	\pause
	\begin{lstlisting}
public static EmployeeAssertion assertThat(Employee employee) {
  return new EmployeeAssertion(employee);
}
	\end{lstlisting}
\end{frame}


\section{Mockito}

\subsection{Mocks}

\begin{frame}{Qu'est-ce qu'un \emph{Mock} ?}

	\begin{description}[<+->]
	\item[Mock] Bouchon dont le comportement peut-être décrit pour chaque test et où les appels peuvent être contrôlés.
	\item[Mocker] (terme non officiel) Remplacer les dépendances d'une brique applicative par des \emph{mocks}.
	\end{description}

\end{frame}

\begin{frame}{Mockito}
	\framesubtitle{\emph{"A mocking framework that tastes really good."}}

	\begin{block}{Mockito}
		\emph{Mockito} est un framework qui isole une brique, et vérifie son comportement vis à vis de ses dépendances. Il propose une API pour valider les appels, et diriger le comportement des autres briques.
	\end{block}
	
\end{frame}

\subsection{Utiliser Mockito}
\begin{frame}[containsverbatim]{Mocker une brique applicative}

	\begin{lstlisting}
public class EmployeeManagerImplTest {

  /** Classe mockee */
  @InjectMock
  private EmployeeManagerImpl employeeManager;
  
  /** Mock du generateur d'email */
  @Mock
  private IEmailGenerator emailGenerator;
  
  /** Mock de la couche de persistance */
  @Mock
  private IEmployeeDAO employeeDao;
  
  /** Espionne/Controle les appels au generateur de matricule */
  @Inject
  @Spy
  private IEmployeeNumberGenerator employeeNumberGenerator;
}
	\end{lstlisting}
	
\end{frame}

\subsection{Assertions}
\begin{frame}[containsverbatim]{Assertions}
	\framesubtitle{Vérifier que les dépendances soient correctement appelées}

	\begin{lstlisting}
@Test
public void test_create_new_employee() {
  // CREATION DU JEU DE TEST
  when(emailGenerator.generateEmail(any(Employee.class))).thenReturn("ironman@stark-enterprise.us");
  when(employeeNumberGenerator.generateNumber()).thenReturn(42, 43, 44);
  
  // EXECUTION
  Employee employee = new Employee("Tony", "Stark");
  employee.setEnterprise("Stark");
  
  employeeManager.createNewEmployee(employee);
  
  // ASSERTIONS
  // mon client est complet
  assertThat(employee).hasFirstnameAndLastName("TONY", "STARK")
  					  .isHiredBy("stark")
  					  .hasEmail("ironman@stark-enterprise.us")
  					  .hasNumber(42);
  
  // les generateurs ont ete correctement appeles et l'employe a ete sauvegarde
  verify(emailGenerator).generateEmail(employee);
  verify(employeeNumberGenerator).generateNumber();
  verify(employeeDao).save(employee);
  
  verifyNoMoreInteractions(employeeDao, employeeNumberGenerator, emailGenerator);
}
\end{lstlisting}
	
\end{frame}

\section{Conclusion}

\begin{frame}{Ce qu'il faut retenir...}
	\framesubtitle{Définitions}

	\begin{description}[<+->]
	\item[Tests unitaires] Isole une brique applicative pour valider qu'elle remplisse son rôle, indépendamment de ses dépendances.
	\item[TDD] \emph{Développement Piloté par les Tests} : commencer à écrire les tests avant de développer la fonctionnalité
	\item[JUNIT] Framework exécutant les tests unitaires
	\item[FestAssert] Facilite l'écriture des assertions
	\item[Mockito] Facilite l'isolation des briques applicatives
	\end{description}
	
\end{frame}

\begin{frame}{Ce qu'il faut retenir...}
	\framesubtitle{Annotations}
	
	Annotations à retenir :
	\begin{description}[<+->]
	\item[\texttt{$@$Test}] Déclare un test sur une méthode public sans paramètres ni retour
	\item[\texttt{$@$InjectMocks}] Remplace les dépendances par des \emph{Mocks}
	\item[\texttt{$@$Mock}] Déclare un attribut comme étant un mock à injecter via l'annotation \texttt{$@$InjectMocks}
	\item[\texttt{$@$Spy}] Espionne les appels effectués sur une brique applicative ; donne la possibilité de l'injecter comme un Mock
	\end{description}

\end{frame}

\begin{frame}{Ce qu'il faut retenir...}
	\framesubtitle{Methodes}
	
	Annotations à retenir :
	\begin{description}[<+->]
	\item[\texttt{assertThat}] Début d'une assertion sur un objet (donné en paramètre)
	\item[\texttt{when}] Défini le comportement d'un \emph{Mock}
	\item[\texttt{verify}] Vérifie les appels qui ont été effectués sur un \emph{Mock}.
	\end{description}

\end{frame}
%Ne remplace pas les tests d'intégration !

\section*{Fin}

\begin{frame}
	\frametitle{Fin}
	\begin{center}
		\huge
		Merci, des questions ?
	\end{center}
\end{frame}

\end{document}
































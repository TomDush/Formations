\documentclass[small,algo]{dushClass}


\title{TP -- Spring}
\subtitle{Initialisation aux Frameworks}

\begin{document}

%%%%%%%%%%%%%%%%%%%%%%%%%%%%%%%%%%%%%%%%%%%%
%% SUJET
\section{Sujet}\label{spec-sec}

\subsection{Objectifs}

Dans ce TP, nous reprendrons \emph{la bibliothèque} du précédent TP afin de lui rajouter une couche métier (business). Il se décomposera en 2 parties : 
\begin{itemize}
\item Séparation des préoccupations et injection de dépendances
\item Coupage de \emph{Spring} à \emph{Hibernate} pour réaliser la couche DAO.
\end{itemize}

\subsection{Modèle de données}

Le modèle de données, présenté figure \ref{model} \vpageref{model}, est le même que dans le TP sur Hibernate (section : \emph{pour aller plus loin}). Lui sont ajoutés l'aspect d'exemplaire (\texttt{BookCopy}), et d'emprunt (\texttt{Customer}).

\begin{figure}[ht]\label{model}
	\center
	\includegraphics{images/model_real.png}
	\caption{Modèle de données}
\end{figure}

\subsection{Cas étudiés}\label{spec-ssec}

Nous allons étudier 2 cas fonctionnels :
\begin{itemize}
\item Création d'une nouvelle référence et ajout d'exemplaires
\item Emprunt d'un exemplaire par un client%si pas déjà 3 locations...
\end{itemize}

\subsubsection{Nouvelle référence}\label{spec-sssec}
L'ajout d'une nouvelle référence et d'exemplaires liés suis la procédure ci-après :
\begin{enumerate}
\item\label{add-ihm} L'administrateur renseigne l'identifiant de la libraire et les données sur le livre qu'il souhaite ajouter (code ISBN du livre, titre, description, nombre de pages), ainsi que le nombre d'exemplaires.
\item Recherche de la librairie par son identifiant, elle doit exister
\item Vérification que la référence n'existe pas déjà (par son code ISBN), sinon la référence est mise à jour
\item Génération d'un code pour chacun des exemplaires
\item Sauvegarde de la référence \texttt{Book}, et des exemplaires, en BDD.\\
\end{enumerate}

Le point \ref{add-ihm} ne fait pas partie du TP. Il sera simulé dans la fonction \texttt{main}.

\subsubsection{Emprunt d'un exemplaire}

Pour emprunter un exemplaire, le processus est :
\begin{enumerate}
\item L'utilisateur indique son code client et le code de l'exemplaire qu'il souhaite emprunter
\item le client et l'exemplaire sont recherchés en base, ils doivent exister.
\item vérifications : l'exemplaire n'est pas déjà emprunté, le client ne doit pas avoir plus de 3 emprunts en cours
\item ajouter l'exemplaire à la liste des emprunt en cours du client\\
\end{enumerate}

Le nombre d'exemplaires autorisés doit être configuré dans un fichier \emph{properties}.

\subsection{Architecture}
L'architecture logicielle suit le principe de la \emph{Séparation des Préoccupations}. Elle est présentée figure \ref{tp-spring-soc} \vpageref{tp-spring-soc}.\\

\begin{figure}[ht]\label{tp-spring-soc}
	\center
	\includegraphics{images/tp-spring-soc.png}
	\caption{Architecture des objets métiers}
\end{figure}

Les briques logicielles sont :
\begin{description}
\item[BookManager] contient les règles métiers pour réaliser les 2 cas étudiés
\item[CopyCodeGenerator] Générateur des codes exemplaires, utilisant le \texttt{BookStoreDAO} pour trouver un nouveau code.
\item[BookDAO] couche d'accès aux données \texttt{Book} : recherche de livre à partir du code isbn, sauvegarde et suppression
\item[BookStoreDAO] couche d'accès aux données \texttt{BookStore} : recherche de la librairie par son nom
\item[BookCopyDAO] couche d'accès aux données \texttt{BookCopy} : génération d'un code unique, recherche d'un copie par son code
\item[CustomerDAO] couche d'accès aux données \texttt{Customer} : recherche par ID.
\end{description}


\subsection{Composants fournis}
Afin de gagner du temps, certains composant ont déjà été créés.

\subsubsection{Contexte Spring}
La configuration principale de Spring est dans le fichier : \texttt{src/main/resources/spring/books-context.xml}.\\

Dans la fonction \texttt{main}, le contexte est déjà instancié et est prêt à l'utilisation comme l'était la \texttt{SessinFactory} d'Hibernate lors du premier TP.

\subsubsection{Mapping relationnel}
Le modèle présenté figure \ref{model} \vpageref{model} est déjà mappé à la base de données. Il vous faudra peut-être supprimer les tables existant déjà dans votre base locale.

\subsubsection{Interfaces}
Les interfaces de chaque brique ont été crées. Comme dans la convention de nommage utilisée dans le Référentiel Coordonnées, elles commencent par un "\emph{I}" ("i" majuscule).


\subsubsection{Couche DAO bouchonnée}
Une fausse implémentation de la couche DAO est présente dans le package \texttt{net.yvesrocher.training.frameworks.dao.mock}. Celle-ci utilise des \texttt{HashMap} en interne pour simuler une véritable base de données.\par
Pour activer cette couche bouchonnée, il faut ajouter le fichier de configuration \emph{Spring} : \texttt{src/main/resources/spring/books-daomock.xml}. Elle est pré-activée dans le socle du TP fourni.\\

Pour remplir la base de données et lister son contenu, la classe \texttt{net.yvesrocher.training.frameworks.dao.utils.BddMonitorImpl} est présente. En la configurant comme un \emph{singleton}, la base sera automatiquement remplie au démarrage du contexte \emph{Spring}.\par
Elle expose une méthode \texttt{printAll()} qui écrit dans la console l'intégralité de la BDD.



%%%%%%%%%%%%%%%%%%%%%%%%%%%%%%%%%%%%%%%%%%%%
%% DI
\section{Première Partie : Injection de Dépendances}

\subsection{Périmètre de la première partie}
Dans cette première partie, nous nous intéresserons au principe d'\emph{Inversion de Contrôle} :
\begin{itemize}
\item Déclarer un \emph{bean}
\item Modifier sa \emph{portée} (ou \emph{scope})
\item Injecter des dépendances\\
\end{itemize}

Nous travaillerons donc uniquement sur la \emph{couche business}, et utiliseront la couche DAO bouchonnée.

\subsection{Déclaration des beans et injection}

\subsubsection{Premier bean}

Dans cette partie :
\begin{enumerate}
\item Créez une implémentation à l'interface \texttt{IBookManager}
\item Déclarez la comme un \emph{Bean}\footnote{Rappel : un \emph{bean} est le nom d'une "brique applicative" pour Spring.}
\item Récupérez une instance à partir du contexte Spring dans la méthode \texttt{main}\\
\end{enumerate}

Vous pouvez récupérer plusieurs \texttt{IBookManager} et vérifier que ce soit la même instance\footnote{Réalisation d'un classe (exemple : voiture). A ne pas confondre avec une classe qui correspondrait aux plans de la voiture.} qui est renvoyée.\par
Modifier ensuite le scope en \texttt{singleton}, puis en \texttt{prototype} pour constater des différences de comportement.

\subsubsection{Première dépendance}

Le \emph{BookManager} ne peut pas travailler seul. Il a besoin du générateur de code d'exemplaire, et de la couche d'accès aux données des livres et des clients.\\

\begin{enumerate}
\item Créez une implémentation de l'interface \texttt{ICopyCodeGenerator}
\item Déclarez la comme un \emph{Bean}
\item Injectez un ICopyCodeGenerator dans le \emph{BookManager}\\
\end{enumerate}

Vérifier qu'en appelant une méthode de \texttt{BookManager}, ce dernier puisse accéder au \texttt{CopyCodeGenerator}.

\subsubsection{Première configuration}
Ajouter un fichier de propriétés : \texttt{src/main/resources/config/books.properties} avec la propriété : \texttt{borrow.limit}. Récupérez cette valeur (en l'injectant) dans le \texttt{BookManager}.

\subsection{Implémenter les beans}

\subsubsection{BddMonitor}

Configurez le \texttt{BddMonitorImpl} pour être un bean de type singleton. Récupérez le dans la méthode \texttt{main} afin d'afficher le contenu de la base de données.


\subsubsection{CopyCodeGenerator}

Implémentez les méthodes du bean \texttt{CopyCodeGeneratorImpl} :
\begin{itemize}
\item appeler le \texttt{CopyBookDAO} pour obtenir un nouveau code
\end{itemize}

\subsubsection{BookManager}
Implémentez la méthode \texttt{BookManagerImpl.insertReference} pour remplir le cahier des charges partie \ref{spec-sec} \ref{spec-ssec} \ref{spec-sssec}.\\

S'il reste du temps, implémenter la méthode \texttt{BookManagerImpl.borrowBook}.

\subsubsection{Tester}

Vérifier le contenu de la base de données avant après pour valider le bon fonctionnement des méthodes.

\subsection{Pour aller plus loin}

Déclarer un autre bean répondant à l'interface \texttt{ICopyCodeGenerator}. Que ce passe-t-il à la création du bean \texttt{BookManager} ?\\

Nommer les générateurs et définissez celui qui doit être injecter dans le \texttt{BookManager}. Récupérez l'autre dans la méthode \texttt{main}.



%%%%%%%%%%%%%%%%%%%%%%%%%%%%%%%%%%%%%%%%%%%%
%% DAO
\section{Seconde Partie : Couche d'Accès aux Données}
% Spring + Hibernate

\subsection{Périmètre de la seconde partie}
Dans cette seconde partie, nous nous intéressons à la couche d'accès au données. Nous sohaitons la connecter à une base de données SQL à l'aide d'\emph{Hibernate}.


\subsection{Configuration du contexte Spring}

Avant tout, il faut configurer Spring pour qu'il puisse injecter la \texttt{SessionFactory}. Un fichier de configuration est prêt : \texttt{src/main/resources/spring/books-persistence.xml}. Ajoutez le à la liste lors de la création du contexte Spring.\\

Les nouveaux beans utilisant hibernate risquent de rentrer en conflit avec les bouchons. Retirez le fichier \texttt{src/main/resources/spring/books-daomock.xml} de la liste des fichiers de configuration.


\subsection{Ré-implémenter la couche DAO}

Implémenter chacune des briques DAO à partir de son interface :
\begin{itemize}
\item IBookCopy
\item IBookDAO
\item IBookStoreDAO
\item ICustomerDAO
\end{itemize}



\subsubsection{Injection de la session factory}

La session factory est maintenant injectable :
\begin{lstlisting}
@Named
public class EmployeeDAOImpl implements IEmployeeDAO {
	
	@Inject
	private SessionFactory factory;
}
\end{lstlisting}


\subsubsection{Ouverture de la session}
La session est ouverte automatiquement lors de l'ouverture d'une transaction. On respecte le pattern : \emph{1 transaction = 1 session}.\\

Rappel, les transaction sont gérées avec l'annotation \texttt{$@$Transactionnal}

% pas dans le RC : open session in view.



%%%%%%%%%%%%%%%%%%%%
% Suite ...
% Cette configuration est faite dans le fichier Spring : \texttt{src/main/resources/spring/books-persistence.xml}.

\end{document}















































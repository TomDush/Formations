\documentclass[compress]{beamer}%
\usepackage[utf8]{inputenc}
\usetheme{Warsaw}


\title{Hibernate}
\subtitle{Utilisation avancées}

\author{Thomas Duchatelle (duchatelle.thomas@gmail.com)}
\institute{Capgemini, pour Yves Rocher}

\setbeamertemplate{navigation symbols}{} 
\useoutertheme{infolines}
\setbeamertemplate{footline}
{%
  \leavevmode%
  \hbox{\begin{beamercolorbox}[wd=.5\paperwidth,ht=2.5ex,dp=1.125ex,leftskip=.3cm plus1fill,rightskip=.3cm]{author in head/foot}%
    \usebeamerfont{author in head/foot}\insertshortauthor
  \end{beamercolorbox}%
  \begin{beamercolorbox}[wd=.41\paperwidth,ht=2.5ex,dp=1.125ex,leftskip=.3cm,rightskip=.3cm plus1fil]{title in head/foot}%
    \usebeamerfont{title in head/foot}\insertshorttitle 
  \end{beamercolorbox}%
  \begin{beamercolorbox}[wd=.09\paperwidth,ht=2.5ex,dp=1.125ex,leftskip=.3cm plus1fill,rightskip=.3cm]{author in head/foot}%
    \usebeamerfont{author in head/foot}\insertframenumber/\inserttotalframenumber
  \end{beamercolorbox}}%
  \vskip0pt%
}

\AtBeginSection[]{
  \begin{frame}{Sommaire}
  \small \tableofcontents[currentsection, hideothersubsections]
  \end{frame} 
}

\definecolor{fontcolor}{rgb}{0.92,0.92,0.99}
\usepackage{listings}
\lstset{language=Java, numbers=left, tabsize=2, frame=single, breaklines=true,  numberstyle=\tiny\ttfamily,basicstyle=\small, framexleftmargin=5mm, backgroundcolor=\color{fontcolor}, xleftmargin=5mm, basicstyle=\tiny }

\graphicspath{images}

\begin{document}


% Pages de présentations...
\frame{\titlepage}
  
\section*{Plan}
\frame{\tableofcontents[hideallsubsections]}
	
%%%%%%%%%%%%%%%%%%%%%%%%%%%%%%%%%%%%%%%%%
%% CONSTAT DES DEGRADATIONS
\section{Comment dégrader les performances avec Hibernate ?}

\begin{frame}{Comment dégrader les performances avec Hibernate ?}
	\framesubtitle{Plus facile qu'il n'y parait !}
	
	\begin{block}{Les opérations secrètes}
	Hibernate réalise beaucoup d'opérations de façon transparente pour nous simplifier la vie. Est-ce que cela nous la simplifie vraiment ?
	\end{block}		
	
	\pause
	Quelques idées pour dégrader les performances : 
	\begin{itemize}[<+->]
	\item Ne jamais fermer la session : plus il y a d'objets dans la session, plus le \texttt{flush} sera long.
	\item Charger toutes les dépendances d'un objet à chaque fois, même si on en a pas besoin
	\item Ne pas vérifier les \texttt{select} générés : les jointures d'Hibernate sont idéales
	\end{itemize}
\end{frame}


\section{Comment l'optimiser ?}

\subsection{Les requêtes SELECT}
\begin{frame}{Les requêtes SELECT}
	
	\begin{block}{HQL : ami ou ennemi}
	Le langage HQL facilite l'écriture des requêtes SQL en éliminant la partie technique, et en axant la requête sur un point de vue \emph{objets}.
	\end{block}
	
	\pause
	\begin{alertblock}{Jointures}
	Mais il ne remplace pas la connaissance SQL ! La facilité d'écriture du HQL cache certaines jointures, il ne faut pas les ignorer.
	\end{alertblock}
	
	\pause
	\begin{block}{Optimisations}
	Une requête HQL s'optimise, comme une requête SQL.
	\end{block}
\end{frame}

\subsection{Chargement des dépendances}
\begin{frame}{Chargement des dépendances}
	\framesubtitle{Chargement feignant (LAZY)}
	
	\begin{block}{LAZY}
	Lorsque le chargement dit \texttt{LAZY} est activé sur une relation, la dépendance n'est chargée (requête en base) que lors de l'accès à la méthode \texttt{get}.
	\end{block}
	
	\pause
	\begin{itemize}
	\item le mode est actif par défaut sur toutes les relations
	\item son contraire est \texttt{EAGER}
	\item l'objet doit être \emph{persistant}\footnote{persistant = rattaché à une session \emph{non fermée}} lors du premier accès à la méthode \texttt{getX}
	\end{itemize}
	
	\pause	
	\begin{block}{Initialisation d'une entité}
	L'initialisation des attributs et collections peut être forcé en appelant la méthode \texttt{Hibernate.initialize(entity.getX());}
	\end{block}
\end{frame}

\subsection{Grouper les requêtes}
\begin{frame}[fragile]{Grouper les requêtes}
	\framesubtitle{Charger les dépendances en une requête}
	
	\begin{exampleblock}{Cas d'exemple}	
	Chargement d'une liste d'employés dont on souhaite, entre autre, connaitre l'employeur.
	\end{exampleblock}
	
	\pause
	Nombre de requêtes \texttt{select} en base :
	\begin{itemize}
	\item 1 sur la table \texttt{employee}
	\item puis 1 par employé sur la table entreprise !
	\end{itemize}

\end{frame}

\begin{frame}[fragile]{Grouper les requêtes}
	\framesubtitle{Forcer la jointure}
	
	\begin{block}{Par requête}	
	L'idée d'optimisation est de forcer la jointure entre l'employé, et l'entreprise :\\
	\begin{lstlisting}
	SELECT e FROM Employee e OUTER JOIN FETCH e.enterprise
	\end{lstlisting}
	\end{block}
	
	\pause
	\begin{block}{Par mapping}	
	L'idée d'optimisation est de forcer la jointure entre l'employé, et l'entreprise :\\
	\begin{lstlisting}
	@Fetch(FetchMode.JOIN)
	public Enterprise getEnterprise() { ... }
	\end{lstlisting}
	\end{block}
	

\end{frame}

\begin{frame}[containsverbatim]{Grouper les requêtes}
	\framesubtitle{Alternatives...}
	\begin{block}{BatchSize}
	Charger simultanément les dépendances de \texttt{n} attribut avec l'annotation \texttt{$@$BatchSize}.
	\end{block}
	
	\begin{lstlisting}
@BatchSize(size=20)
public class Employee ... {
	...
	
	@BatchSize(size=20)
	public Enterprise getEnterprise() { ... }
}
	\end{lstlisting}
	
\end{frame}


\subsection{Cascades}
\begin{frame}{Les cascades}
	\framesubtitle{Ici le Niagara}

	\begin{block}{Cascade}
	Le paramètre cascade est présent sur toutes les associations. Il définie le comportement de la session vis à vis des dépendances.
	\end{block}
	
	\pause
	Les principales valeurs possibles :
	\begin{itemize}
	\item \texttt{PERSIST} : les dépendances sont sauvegardées si l'entité l'est
	\item \texttt{REMOVE} : les dépendances sont supprimées avec l'entité
	\item \texttt{ALL} : toutes les actions sur l'entité sont répercutées sur les dépendances.
	\end{itemize}
\end{frame}


\section*{Fin}

\begin{frame}
	\frametitle{Fin}
	\begin{center}
		\huge
		Merci, des questions ?
	\end{center}
\end{frame}


\end{document}
